\documentclass{article}


\usepackage[utf8x]{inputenc}
\usepackage[T1]{fontenc}
\usepackage[frenchb]{babel}
\usepackage{couvresumesPFE}
\usepackage{eurosym}
\usepackage[table]{xcolor}
\usepackage{booktabs}

\ordre{}
\auteur{Victor}{Gaudard}
\sautverticalnegatif{1.4}
\specialite{INFO}
\anneeuniversitaire{2016 --- 2017}
\titre{}
\entreprisenom{IT Link}
\tuteur{Grégory}{Chagnet}
\correspondantINSA{Bertrand}{Couasnon}
\entrepriselogo{IT Link}{logo_itlink}{4 cm}

\resumefrancais{Résumé TODO}
\resumeanglais{Abstract TODO}

\begin{document}

\makepfecouverture{insa}

\newpage

\section*{Introduction}

Ce rapport porte sur mon stage de trois mois durant l'été 2017 (du 3
juillet au 29 septembre 2017) dans le cade de ma formation d'élève
ingénieur à l'INSA.

\section*{Remerciements}

\newpage

\tableofcontents

\newpage

\section{Présentation de l'entreprise}

\subsection{IPSIS et Groupe IT Link}

IPSIS est une société qui a rejoint et créé le groupe IT Link en se
joignant à IT Link et IT Link System en 2000. Le groupe IT Link est une
ESN\footnote{ESN : Entreprise de service numérique}.

Le groupe IT Link a une histoire dans la conception de systèmes
embarqués pour l'automobile et les systèmes critiques pour la Défense.
Ainsi, le groupe a su capitaliser une expérience en
R\&D\footnote{R\&D : Recherche et développement} et en industrialisation pour le
compte de tiers. Cette expérience permet au groupe d'anticiper et de
satisfaire les exigences de clients d'autres secteurs économiques. Le
groupe IT Link est présent dans 9 agences en France, en plus d'agences
en Belgique, en Allemagne et au Canada. L'équipe belge est spécialisée
dans la conception matérielle technologique et de logiciel embarqué.
L'équipe allemande est rattachée au Centre National de Compétence
Véhicule de la région Alsace et Franche-Comté. Elle intervient auprès de
constructeurs et équipementiers allemands. L'équipe allemande travaille
dans les secteurs de transports, d'aéronautique, d'énergie, de
TIC\footnote{TIC : Technologies de l'information et de la communication}.

Quelques chiffres :

\begin{itemize}
    \item Chiffre d'affaires en 2016 : 42M\euro{}
    \item 30 ans d'existence
    \item 550 salariés donc plus de 90\% d'ingénieurs
    \item 16 implantations dont 12 en France et 4 à l'international
\end{itemize}

\subsection{IPSIS Rennes}

L'agence de Rennes est un des deux grands centres de production du
groupe IT Link. Ce centre concentre des expertises méthodologiques pour
développer des systèmes logiciels.

Elle compte environ 200 consultants dont 40 sur le site, travaillant
dans des domaines tels que :
\begin{itemize}
    \item Les capteurs et l'intelligence embarquée
    \item La télécommunication
    \item La cybersécurité
    \item Les applications mobiles
    \item Le Big Data
    \item Le web
    \item Les objets connectés
    \item Les systèmes connectés industriels
\end{itemize}

IPSIS réalise également de la R\&D dans la modélisation et le contrôle
de systèmes dynamiques, dans la conception de systèmes de communication
telle que la radiocommunication et l'optique.

L'agence a su s'intégrer en Bretagne en s'adaptant aux activités locales
tels que la défense, le développement durable, la gestion des risques,
la cybersécurité et la domotique. Elle entretient également de bons
échanges avec les écoles majeures de la région avec notamment un
partenariat avec l'INSA de Rennes.

IPSIS réalise des études au forfait et de l'assistance technique chez
ses clients. Elle conseille, expertise et forme dans les domaines
scientifiques et techniques. Non seulement l'entreprise développe des
logiciels, des techniques et des services innovants, mais elle fait
aussi l'effort de développer ses propres applications et prototypes dans
le but de rester à la pointe de la technologie et de proposer à ses
clients des solutions supplémentaires.

\newpage

\section{Contexte du projet}

\section{Réalisation du projet}

\section{Conclusion}

\newpage
\appendix

\newpage

\makepfequatriemeinsa{}


\end{document}
